%\newcounter{MYtempeqncnt}
\DeclareMathOperator*{\Res}{Res}
%\renewcommand{\baselinestretch}{2}
\iffalse
\renewcommand\thesection{\arabic{section}}
\renewcommand\thesubsection{\thesection.\arabic{subsection}}
\renewcommand\thesubsubsection{\thesubsection.\arabic{subsubsection}}

\renewcommand\thesectiondis{\arabic{section}}
\renewcommand\thesubsectiondis{\thesectiondis.\arabic{subsection}}
\renewcommand\thesubsubsectiondis{\thesubsectiondis.\arabic{subsubsection}}
\fi

% correct bad hyphenation here
\hyphenation{op-tical net-works semi-conduc-tor}
\def\inputGnumericTable{}                                 %%

\lstset{
%language=C,
frame=single, 
breaklines=true,
columns=fullflexible
}
%\lstset{
%language=tex,
%frame=single, 
%breaklines=true
%}


\providecommand{\pr}[1]{\ensuremath{\Pr\left(#1\right)}}
\providecommand{\prt}[2]{\ensuremath{p_{#1}^{\left(#2\right)} }}        % own macro for this question
\providecommand{\qfunc}[1]{\ensuremath{Q\left(#1\right)}}
\providecommand{\sbrak}[1]{\ensuremath{{}\left[#1\right]}}
\providecommand{\lsbrak}[1]{\ensuremath{{}\left[#1\right.}}
\providecommand{\rsbrak}[1]{\ensuremath{{}\left.#1\right]}}
\providecommand{\brak}[1]{\ensuremath{\left(#1\right)}}
\providecommand{\lbrak}[1]{\ensuremath{\left(#1\right.}}
\providecommand{\rbrak}[1]{\ensuremath{\left.#1\right)}}
\providecommand{\cbrak}[1]{\ensuremath{\left\{#1\right\}}}
\providecommand{\lcbrak}[1]{\ensuremath{\left\{#1\right.}}
\providecommand{\rcbrak}[1]{\ensuremath{\left.#1\right\}}}
\newcommand{\sgn}{\mathop{\mathrm{sgn}}}
\providecommand{\abs}[1]{\left\vert#1\right\vert}
\providecommand{\res}[1]{\Res\displaylimits_{#1}} 
\providecommand{\norm}[1]{\left\lVert#1\right\rVert}
%\providecommand{\norm}[1]{\lVert#1\rVert}
\providecommand{\mtx}[1]{\mathbf{#1}}
\providecommand{\mean}[1]{E\left[ #1 \right]}
\providecommand{\cond}[2]{#1\middle|#2}
\providecommand{\fourier}{\overset{\mathcal{F}}{ \rightleftharpoons}}
\newenvironment{amatrix}[1]{%
  \left(\begin{array}{@{}*{#1}{c}|c@{}}
}{%
  \end{array}\right)
}
%\providecommand{\hilbert}{\overset{\mathcal{H}}{ \rightleftharpoons}}
%\providecommand{\system}{\overset{\mathcal{H}}{ \longleftrightarrow}}
	%\newcommand{\solution}[2]{\textbf{Solution:}{#1}}
\newcommand{\solution}{\noindent \textbf{Solution: }}
\newcommand{\cosec}{\,\text{cosec}\,}
\providecommand{\dec}[2]{\ensuremath{\overset{#1}{\underset{#2}{\gtrless}}}}
\newcommand{\myvec}[1]{\ensuremath{\begin{pmatrix}#1\end{pmatrix}}}
\newcommand{\mydet}[1]{\ensuremath{\begin{vmatrix}#1\end{vmatrix}}}
\newcommand{\myaugvec}[2]{\ensuremath{\begin{amatrix}{#1}#2\end{amatrix}}}
\providecommand{\rank}{\text{rank}}
\providecommand{\pr}[1]{\ensuremath{\Pr\left(#1\right)}}
\providecommand{\qfunc}[1]{\ensuremath{Q\left(#1\right)}}
	\newcommand*{\permcomb}[4][0mu]{{{}^{#3}\mkern#1#2_{#4}}}
\newcommand*{\perm}[1][-3mu]{\permcomb[#1]{P}}
\newcommand*{\comb}[1][-1mu]{\permcomb[#1]{C}}
\providecommand{\qfunc}[1]{\ensuremath{Q\left(#1\right)}}
\providecommand{\gauss}[2]{\mathcal{N}\ensuremath{\left(#1,#2\right)}}
\providecommand{\diff}[2]{\ensuremath{\frac{d{#1}}{d{#2}}}}
\providecommand{\myceil}[1]{\left \lceil #1 \right \rceil }
\newcommand\figref{Fig.~\ref}
\newcommand\tabref{Table~\ref}
\newcommand{\sinc}{\,\text{sinc}\,}
\newcommand{\rect}{\,\text{rect}\,}
%%
%	%\newcommand{\solution}[2]{\textbf{Solution:}{#1}}
%\newcommand{\solution}{\noindent \textbf{Solution: }}
%\newcommand{\cosec}{\,\text{cosec}\,}
%\numberwithin{equation}{section}
%\numberwithin{equation}{subsection}
%\numberwithin{problem}{section}
%\numberwithin{definition}{section}
%\makeatletter
%\@addtoreset{figure}{problem}
%\makeatother

%\let\StandardTheFigure\thefigure
\let\vec\mathbf

\NewDocumentCommand{\system}{gg}{%
  \IfNoValueTF{#1}
    {%
 \longleftrightarrow
    }
    {%
      \IfNoValueTF{#2}
      {%
\overset{\mathcal{#1}}{ \longleftrightarrow}
      }
      {%
\overset{\mathcal{#1}^{#2}}{ \longleftrightarrow}
      }
    }%
}
%\numberwithin{equation}{section}
%\numberwithin{equation}{subsection}
%\numberwithin{problem}{section}
%\numberwithin{definition}{section}
%\makeatletter
%\@addtoreset{figure}{problem}
%\makeatother

%\let\StandardTheFigure\thefigure
\let\vec\mathbf
%\renewcommand{\thefigure}{\theproblem.\arabic{figure}}
%\renewcommand{\thefigure}{\theproblem}
%\setlist[enumerate,1]{before=\renewcommand\theequation{\theenumi.\arabic{equation}}
%\counterwithin{equation}{enumi}


%\renewcommand{\theequation}{\arabic{subsection}.\arabic{equation}}

%\def\putbox#1#2#3{\makebox[0in][l]{\makebox[#1][l]{}\raisebox{\baselineskip}[0in][0in]{\raisebox{#2}[0in][0in]{#3}}}}
%     \def\rightbox#1{\makebox[0in][r]{#1}}
%     \def\centbox#1{\makebox[0in]{#1}}
%     \def\topbox#1{\raisebox{-\baselineskip}[0in][0in]{#1}}
%     \def\midbox#1{\raisebox{-0.5\baselineskip}[0in][0in]{#1}}
\newcounter{matchleft}\newcounter{matchright}

\newenvironment{matchtabular}{%
  \setcounter{matchleft}{0}%
  \setcounter{matchright}{0}%
  \tabularx{\textwidth}{%
    >{\leavevmode\hbox to 1.5em{\stepcounter{matchleft}\arabic{matchleft}.}}X%
\begin{enumerate}[label=\thechapter.\arabic*,ref=\thechapter.\theenumi]
\numberwithin{equation}{enumi}
\numberwithin{figure}{enumi}
\numberwithin{table}{enumi}
\item There are 5\% defective items in a large bulk of items. What is the probability
that a sample of 10 items will include not more than one defective item?
\item A person buys a lottery ticket in 50 lotteries, in each of which his chance of
winning a prize is $\frac{1}{100}$ . What is the probability that he will win a prize
(a) at least once (b) exactly once (c) at least twice?
\item If 4-digit numbers greater than 5,000 are randomly formed from the digits
0, 1, 3, 5, and 7, what is the probability of forming a number divisible by 5 when,
(i) the digits are repeated? (ii) the repetition of digits is not allowed?
\item A fair coin and an unbiased die are tossed. Let A be the event `head appears on the coin' and B be the event `3 on the die'. Check whether A and B are independent events or not.
\item It is known that 10\% of certain articles manufactured are defective. What is the probability that in a random sample of 12 such articles, 9 are defective?
\item Suppose $X$ has a binomial distribution $B\brak{6,\frac{1}{2}}$. Show that $X = 3$ is the most likely outcome. 
(Hint : P$\brak{X = 3}$ is the maximum among all P$\brak{x_i}, x_i = 0,1,2,3,4,5,6$)
\item In an examination, 20 questions of true-false type are asked. Suppose a student
	tosses a fair coin to determine his answer to each question. If the coin falls
	heads, he answers `true'; if it falls tails, he answers `false'. Find the probability
	that he answers at least 12 questions correctly.
Two Coins are tossed once, where
\begin{enumerate}
\item  E : Tail appears on one coin,\qquad F : one coin shows head
\item  E : no tail appears,\qquad\qquad\qquad F : no head appears
\end{enumerate}
Determine \pr{E\mid F}.
\item A card is selected from a pack of 52 cards.
\begin{enumerate}
\item How many points are there in the sample space?
\item Calculate the probability that the card is an ace of spades.
\item Calculate the probability that the card is 
           \begin{enumerate}
           \item an ace
           \item black card.
           \end{enumerate}
\end{enumerate}
\item Bag I contains 3 red and 4 black balls and Bag II contains 4 red and 5 black balls.
One ball is transferred from Bag I to Bag II and then a ball is drawn from Bag II.
The ball so drawn is found to be red in colour. Find the probability that the
transferred ball is black.
\item How many times must a man toss a fair coin so that the probability of having
at least one head is more than 90\%?
\item In a hurdle race, a player has to cross 10 hurdles. The probability that he will clear each hurdle is $\frac{5}{6}$. What is the probability that he will knock down fewer than 2 hurdles?
\item Three letters are dictated to three persons and an envelope is addressed to each
of them, the letters are inserted into the envelopes at random so that each envelope
contains exactly one letter. Find the probability that at least one letter is in its
proper envelope.
\item In a game, a man wins a rupee for a six and loses a rupee for any other number when a fair dice is thrown.The man decided to throw a dice thrice but to quit as and when he gets a six.Find the expected value of the amount he wins/loses.
\item Two customers Shyam and Ekta are visiting a particular shop in the same week(Tuesday to Saturday). Each is equally likely to visit the shop on any day as on another day. What is the probability that both will visit the shop on (i) the same day? (ii) consecutive days? (iii) different days?
\item Consider the experiment of throwing a die, if a multiple of 3 comes up, throw the die again and if any other number comes, toss a coin. Find the conditional probability of the event 'the coin shows a tail',given that 'atleast one die shows a 3'.
\item A die is thrown again and again until three sixes are obtained. Find the probability of obtaining the third six in the sixth throw of the die.
\item Let X represent the difference between the number of heads and the number of
tails obtained when a coin is tossed 6 times. What are possible values of X?
\end{enumerate}
